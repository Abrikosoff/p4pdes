\chapter{7. Systems of PDEs, and the Stokes equations}

\section{First-order systems}

FIXME: FOSLS for Poisson

\section{Stokes equations}

FIXME: lay out theory from incompressible variable-viscosity Navier-Stokes,
\begin{align*}
\rho \left(\frac{\partial \bu}{\partial t} + \bu \cdot \grad \bu\right) &= \Div \sigma + \bg \\
\sigma &= 2 \nu\, Du - p I \\
\Div \bu &= 0
\end{align*}
where
  $$\grad \bu = \begin{bmatrix}
    \frac{\partial u_1}{\partial x} & \frac{\partial u_2}{\partial x} & \frac{\partial u_3}{\partial x} \\
    \frac{\partial u_1}{\partial y} & \frac{\partial u_2}{\partial y} & \frac{\partial u_3}{\partial y} \\
    \frac{\partial u_1}{\partial z} & \frac{\partial u_2}{\partial z} & \frac{\partial u_3}{\partial z}
    \end{bmatrix}$$
and
    $$D u = \frac{1}{2} \left(\grad \bu + \grad \bu^{\top}\right)$$
to (Froude zero) variable-viscosity Stokes in terms of velocity,
\begin{align*}
\Div \left(\nu \left(\grad \bu + \grad \bu^{\top}\right)\right) - \grad p &= - \bg \\
\Div \bu &= 0
\end{align*}

Consider the case where the viscosity $\nu$ is constant.  Note that by commuting mixed derivatives and using incompressibility we have
\begin{align*}
\Div \left(\grad \bu^{\top}\right) &= \begin{bmatrix}
    \frac{\partial}{\partial x} & \frac{\partial}{\partial y} & \frac{\partial}{\partial z}
    \end{bmatrix}
    \begin{bmatrix}
    \frac{\partial u_1}{\partial x} & \frac{\partial u_1}{\partial y} & \frac{\partial u_1}{\partial z} \\
    \frac{\partial u_2}{\partial x} & \frac{\partial u_2}{\partial y} & \frac{\partial u_2}{\partial z} \\
    \frac{\partial u_3}{\partial x} & \frac{\partial u_3}{\partial y} & \frac{\partial u_3}{\partial z}
    \end{bmatrix} \\
  &= \begin{bmatrix}
    \frac{\partial}{\partial x}\left(\Div \bu\right) & \frac{\partial}{\partial y }\left(\Div \bu\right) & \frac{\partial}{\partial x} \left(\Div \bu\right)
    \end{bmatrix}
    = \begin{bmatrix} 0 & 0 & 0 \end{bmatrix}.
\end{align*}
On the other hand,
\begin{align*}
\Div \left(\grad \bu^{\top}\right) &= \begin{bmatrix}
    \frac{\partial}{\partial x} & \frac{\partial}{\partial y} & \frac{\partial}{\partial z}
    \end{bmatrix}
    \begin{bmatrix}
    \frac{\partial u_1}{\partial x} & \frac{\partial u_2}{\partial x} & \frac{\partial u_3}{\partial x} \\
    \frac{\partial u_1}{\partial y} & \frac{\partial u_2}{\partial y} & \frac{\partial u_3}{\partial y} \\
    \frac{\partial u_1}{\partial z} & \frac{\partial u_2}{\partial z} & \frac{\partial u_3}{\partial z}
    \end{bmatrix}
    = \begin{bmatrix} \grad^2 u_1 & \grad^2 u_2 & \grad^2 u_3 \end{bmatrix}
\end{align*}
which we write as ``$\grad^2 \bu$.''  Thus in the constant viscosity case the Stokes equations are
\begin{align}
\nu \grad^2 \bu - \grad p &= - \bg \\
\Div \bu &= 0
\end{align}

\section{A planar-flow problem}

ours in 2D with gravity body force
\begin{align}
\nu \grad^2 u - p_x &= - g_1 \label{stokes2du} \\
\nu \grad^2 v - p_y &= - g_2 \label{stokes2dv} \\
u_x + u_y &= 0 \label{stokes2dincomp} 
\end{align}
where
    $$\bg = (\rho g \sin \theta, - \rho g \cos \theta) = (g_1,g_2)$$

FIXME: particular problem is flow down a slope which is periodic in $x$

FIXME: particular problem has stress free top
\begin{align*}
\sigma\cdot\bn &= 0 \\
\iff \qquad \begin{bmatrix}
2 \nu u_x - p & \nu (u_y+v_x) \\
\nu (u_y+v_x) & 2 \nu v_y - p
\end{bmatrix} \begin{bmatrix}
g_2 \\ -g_1
\end{bmatrix} &= \begin{bmatrix}
0 \\ 0
\end{bmatrix} \\
\iff \qquad FIXME
\end{align*}

and no-slip bottom, thus these boundary conditions,
\begin{align}
\text{top:}&    & &\begin{array}{l} u_y + v_x = 0 \\ 2 \nu v_y = p\end{array} \\
\text{bottom:}& & &\begin{array}{l} u = 0 \\ v = 0 \end{array}
\end{align}

\section{A finite difference, structured-grid approach}

FIXME: Laplacians are something we understand, so lets proceed with eqns \eqref{stokes2du}, \eqref{stokes2dv}; use FD centered

FIXME: any way we do \eqref{stokes2dincomp} we get block structure for the full system $A \bz = \bbf$; informally the structure is
  $$\begin{bmatrix}
    \nu\grad^2 & 0 & -\grad \\
    0 & \nu\grad^2 & -\grad \\
    \grad & \grad & 0
    \end{bmatrix}
    \begin{bmatrix}
    \bu \\ \bv \\ \bp
    \end{bmatrix}
    =
    \begin{bmatrix}
    - g_1 \\ - g_2 \\ 0
    \end{bmatrix}
    $$
but let's write
  $$\begin{bmatrix}
    L & 0 & G_1 \\
    0 & L & G_2 \\
    B_1 & B_2 & 0
    \end{bmatrix}
    \begin{bmatrix}
    \bu \\ \bv \\ \bp
    \end{bmatrix}
    =
    \begin{bmatrix}
    - g_1 \\ - g_2 \\ 0
    \end{bmatrix}
    $$

FIXME: $A$ certainly not invertible if approx to ``$\Div \bu$'' (i.e.~$\begin{bmatrix} B_1 & B_2 \end{bmatrix}$) has more rows than columns

FIXME: extract approx to ``$\Div \bu$'' as transpose of the ``$\grad p$'' part (i.e.~$\begin{bmatrix} G_1 \\ G_2 \end{bmatrix}$); now system is symmetric:
    $$\begin{bmatrix}
    L & 0 & G_1 \\
    0 & L & G_2 \\
    G_1^\top & G_2^\top & 0
    \end{bmatrix}
    \begin{bmatrix}
    \bu \\ \bv \\ \bp
    \end{bmatrix}
    =
    \begin{bmatrix}
    - g_1 \\ - g_2 \\ 0
    \end{bmatrix}
    \quad \iff \quad A\bz = \bbf
    $$

FIXME: not the best way but illuminating; code uses DMDA and SNES

\section{Preconditioners for indefinite systems}

FIXME: while $A$ is symmetric, it is not positive definite; note that if
    $$M = \begin{bmatrix} 0 & G \\ G^\top & 0 \end{bmatrix}$$
and if $M$ has eigenvalue $\lambda$ so that
    $$M \begin{bmatrix} \bx \\ \by \end{bmatrix} = \lambda \begin{bmatrix} \bx \\ \by \end{bmatrix}$$
then
    $$M \begin{bmatrix} -\bx \\ \by \end{bmatrix} = - \lambda \begin{bmatrix} -\bx \\ \by \end{bmatrix}$$
and (w.o.l.o.g.) this is a lin.-indep. eigenvector so $M$ also has eigenvalue $-\lambda$



\caveat{But \PETSc should help with unstructured meshes too.}



\chapter{4. Multigrid for linear problems}

We now make a key transition from Chapters 2 and 3.  Instead of choosing a grid (at runtime, of course) and writing code to build a matrix form of the problem, on that grid, and handing it to \pKSP for solution, we have \emph{hand a matrix-building code to} \pKSP.  Then the \pKSP can choose the grid.  That is, we show the problem to \PETSc in a way that allows it to do multigrid.

\cinputpart{c4poisson.c}{Poisson problem on a structured grid again, but this time using \texttt{KSPSetComputeOperators()} so that multigrid is possible.}{I}{//MAIN}{//ENDMAIN}{code:multigridone}

\cinputpart{c4poisson.c}{This method builds the right-hand side.}{II}{//COMPUTERHS}{//ENDCOMPUTERHS}{code:multigridtwo}

\cinputpart{c4poisson.c}{This is the method handed to \texttt{KSPSetComputeOperators()}.}{III}{//COMPUTEJAC}{//ENDCOMPUTEJAC}{code:multigridthree}

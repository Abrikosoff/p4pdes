\renewcommand{\CODELOC}{c/ch4/}

\chapter{Nonlinear equations}
\label{chap:nonlinear}


How should nonlinear equations first appear in a code which uses \PETSc?  The answer is that they merely change the functional form of the residual.  For a linear system the residual is this function of the unknowns,
\begin{equation}
\br = \bF(\bu) = \bb - A \bu, \label{eq:nl:linres}
\end{equation}
but we now consider cases in which $\bF$ is a higher-order polynomial, a transcendental function, or some more general function.  In fact, we only suppose $\bF : \RR^N \to \RR^N$ is differentiable.

The input $\bx$ and output $\bF(\bx)$ are column vectors,\sidenote{The name change of the unknown $\bu\to\bx$, relative to \eqref{eq:nl:linres}, is because we will now think more geometrically about changes in the location of $\bx$.} so in that sense $\bF$ acts like multiplying by a square matrix $\bx\mapsto A\bx$.  Just as we would reduce the residual to zero in an iterative linear algebra method, for nonlinear $\bF$ we want to solve
\begin{equation}
   \bF(\bx) = 0   \label{eq:nl:equation}
\end{equation}
by iteration.

In this section we follow Isaac Newton in building an iterative method to solve \eqref{eq:nl:equation}.  This method linearizes \eqref{eq:nl:equation} around the most recent iterate and then ``moves'' $\bx$ to the location which solves the linear problem and is, we hope, closer to the solution.  Each iteration therefore solves a linear system; we already have some \PETSc technology for that!  However, the cost of performing the linearization must be taken into account, choosing a ``smart'' distance to move will require additional choices, and all existing issues regarding the linear solver, including preconditioner choices, as discussed in the last two chapters, remain active.

Large systems of nonlinear equations arise in applications because there are many PDE-type physical models with nonlinearities.  Most of this section is about finite-dimensional systems of nonlinear equations \eqref{eq:nl:equation}, as problems of their own, but discretizing a nonlinear PDE leads to such systems.  At the end of this section we give one example of a discretized one-dimensional nonlinear PDE, and in the next Chapter we continue with another nonlinear PDE problem in two dimensions.


\section{Newton's method}

First we recall Newton's method.  If we have already determined a \emph{step} $\bs$ away from the current iterate $\bx_k$, where both $\bs$ and $\bx_k$ are vectors in $\RR^N$, then $\bx_{k+1} = \bx_k + \bs$ would be the next iterate.  By definition, because $\bF$ is differentiable,
\begin{equation}
    \bF(\bx_{k+1}) = \bF(\bx_k) + J_\bF(\bx_k) \bs + o(\|\bs\|)  \label{eq:nl:expandF}
\end{equation}
for some square matrix
\begin{equation}
J_\bF(\bx_k) = \begin{bmatrix}
    \frac{\partial F_0}{\partial x_0} & \dots & \frac{\partial F_0}{\partial x_{N-1}} \\
    \vdots & \ddots & \vdots \\
    \frac{\partial F_{N-1}}{\partial x_0} & \dots & \frac{\partial F_{N-1}}{\partial x_{N-1}}  \end{bmatrix},  \label{eq:nl:jacdefn}
\end{equation}
and some quantity $o(\|\bs\|)$ that goes to zero as the length of the \emph{step length} $\|\bs\|$ goes to zero.  The matrix $J = J_\bF(\bx_k)$ is called the \emph{Jacobian} of $\bF$ at $\bx_k$, though we also refer to the function $\bx \mapsto J_\bF(\bx)$ as the Jacobian.

An iteration of Newton's method approximately solves nonlinear equation \eqref{eq:nl:equation} by truncating \eqref{eq:nl:expandF} and then seeking $\bs$ so that the updated value $\bF(\bx_{k+1})$ is zero.  That is, each Newton step computes $\bs$ by the linear equation
\begin{equation}
    0 = \bF(\bx_k) + J_\bF(\bx_k) \bs.
\end{equation}
Writing this equation in our previous style ``$A\bu=\bb$'' for linear systems, at each iteration $k$ we solve a linear system and then do a vector addition:
\begin{align}
    J_\bF(\bx_k) \bs &= - \bF(\bx_k)  \label{eq:nl:newtoneq}  \\
    \bx_{k+1} &= \bx_k + \bs  \label{eq:nl:newtonupdate}
\end{align}
This is Newton's method.  It is fairly simple in theory.

Actual practice is not that much more complicated, especially with \PETSc in hand.  Despite the reputation of Newton iteration as fragile or scary, by using a bit of caution in ``moving'' to the new iterate as in \eqref{eq:nl:newtonupdate}, this will work out just fine on many small nonlinear systems \citep{Kelley2003}.  On big nonlinear systems we will need to pay more attention to the details of the linear solve at each step.\sidenote{It is important to emphasize that the nonlinear problem requires all of the tools for linear systems already considered in Chapters \ref{chap:linearsystem} and \ref{chap:structured}, and more.}  In either case, Newton iteration is the core technology for solving nonlinear problems, including nonlinear PDEs.

\vfill
\clearpage
Noting that nonlinear systems can be visualized as the intersections of curves, surfaces, or hypersurfaces, depending on dimension, a small example gives us a start on exploring the details.

\medskip\noindent\hrulefill
\begin{example}  Given $b > 1$ this pair of nonlinear equations
    $$y = \frac{1}{b} e^{bx}, \qquad x^2+y^2 = 1,$$
form intersecting curves in the plane.  As shown in Figure \ref{fig:expcirclebasic} for the $b=2$ case, the curves intersect twice, once each in the first and second quadrants.

\begin{marginfigure}
\includegraphics[width=1.2\textwidth]{expcirclebasic}
\caption{Newton iterates approach a solution of $\bF(\bx)=0$ for $\bF$ in \eqref{eq:nl:expcircleF} and $b=2$.}
\label{fig:expcirclebasic}
\end{marginfigure}

These equations are put in standard form \eqref{eq:nl:equation} by writing
\begin{equation}
\label{eq:nl:expcircleF}
\bF(\bx) = \begin{bmatrix}
           \frac{1}{b} e^{b x_0} - x_1 \\
           x_0^2 + x_1^2 - 1
           \end{bmatrix}
\end{equation}
for $\bx\in \RR^2$ with components $\bx = [x_0 \, x_1]^\top$.  Thus
\begin{equation}
\label{eq:nl:expcircleJac}
J_\bF(\bx) = \begin{bmatrix}
    e^{b x_0} & & -1 \\
    2 x_0   & & 2 x_1 \end{bmatrix}
\end{equation}
Let $b=2$.  If we start the Newton iteration with $\bx_0 = [1 \,\, 1]^\top$ then the sequence of iterates from \eqref{eq:nl:newtoneq} and \eqref{eq:nl:newtonupdate} is
    $$\twovect{\bx}{0}{1}{1}, \quad \twovect{\bx}{1}{0.619203}{0.880797}, \quad \twovect{\bx}{2}{0.394157}{0.948623}, \quad \dots$$
as also shown in Figure \ref{fig:expcirclebasic}.

\noindent\hrulefill
%  FROM $ for N in 0 1 2; do ./expcircle -snes_fd -snes_max_it $N; done
\end{example}


\section{\pSNES and call-backs}

In \PETSc we will do Newton iteration for the above example by using a nonlinear-solver object of type \pSNES.\sidenote{This acronym stands for ``scalable nonlinear equation solver.''}  A \pSNES object has the usual \texttt{Create/SetFromOptions/Destroy} sequence, but it also has a method by which we tell it about the function $\bF$.  This is a ``call-back'' in the sense that we supply a function, here named \texttt{FormFunction()}, which the \pSNES can call.  The \pSNES calls \texttt{FormFunction()} with an argument $\bx$, when it needs $\bF(\bx)$ during the Newton iteration.  Later we will also provide the \pSNES with a function which computes the derivative of $\bF$, the Jacobian function $J_{\bF}$.  However, because this derivative can instead be approximated by finite differences, our first code avoids a Jacobian ``call-back,''

Figure \ref{code:expcircle} shows our entire first \pSNES-using code \texttt{expcircle.c}.  It solves problem \eqref{eq:nl:equation} with $\bF$ from \eqref{eq:nl:expcircleF} and $b=2$.

\vfill
\cinput{expcircle.c}{\CODELOC}{A first \pSNES-using code.  Solves nonlinear system \eqref{eq:nl:equation} with $\bF$ given in \eqref{eq:nl:expcircleF}.}{//START}{//END}{code:expcircle}

The \texttt{main()} portion is mostly not surprising, but we summarize the contents anyway:  We start by allocating \pVec \texttt{x} of fixed dimension 2.  This holds both the initial iterate $\bx_0$ and, once the Newton iteration is ended, the converged estimate of the solution to system \eqref{eq:nl:equation}.  Because both components of $\bx_0$ are $1$ in the above example, \pVec \texttt{x} can be initialized with \texttt{VecSet()}.  Next a duplicate \pVec \texttt{r} is created; the \pSNES will use it as space for the (nonlinear) residual.  Then the \pSNES is created and configured, and, in particular, the formula \eqref{eq:nl:expcircleF} is supplied by a call to \texttt{SNESSetFunction()}.  This ``call-back'' sets the third argument of \texttt{SNESSetFunction()} to the name of the C function \texttt{FormFunction()}, which also appears in Figure \ref{code:expcircle}.  Note \texttt{SNESSetFromOptions()} gives us run-time control both on how the Jacobian is calculated\sidenote{Options \texttt{-snes\_fd} and \texttt{-snes\_mf} are allowed; see below.} and on how the length of the step $\bs$ is actually determined.\sidenote{Through \texttt{-snes\_linesearch\_type} and related options; see below.}  Then the system is solved by a call to \texttt{SNESSolve()}, which also supplies \pVec \texttt{x}.  Finally the new state of \texttt{x}, presumably the converged solution, is printed at the command line using \texttt{VecView()} with a \texttt{STDOUT} viewer.

In order to match the calling sequence of \texttt{SNESSetFunction()}, our \texttt{FormFunction()} must have a particular ``signature'' as a C function:
\begin{code}
PetscErrorCode (*f)(SNES,Vec,Vec,void*)
\end{code}
In particular, \texttt{FormFunction()} takes the input $\bx$ as the first \pVec and it must generate output $\bF(\bx)$ as the second \pVec.  There may be additional information, such as parameters, passed to \texttt{FormFunction()} in a ``user context'' which is the pointer \texttt{void*}; we will return to passing parameters below.

Looking inside \texttt{FormFunction()}, in Figure \ref{code:expcircle}, we see new methods for extracting values from, and setting values in, a \pVec.  Previously we have used \texttt{VecSetValues()} to set values at given indices, but here we access the C array underlying the \pVec instead.  We need to read entries of input \pVec \texttt{x} and then set entries of output \pVec \texttt{F}.  For the former we use the read-only array access method \texttt{VecGetArrayRead()} which supplies us with a read-only pointer \texttt{const PetscReal *ax}.\sidenote{Because of the \texttt{const} qualifier the C compiler can stop us from altering \pVec \texttt{x}.  Try it!}  For example, \texttt{ax[0]} is the first entry of \pVec \texttt{x}.  Since we are setting entries of \pVec \texttt{F}, we do nearly the same but using \texttt{PetscReal *aF} and method \texttt{VecGetArray()}.

To avoid conflicts with other methods reading or writing the same memory, \texttt{VecGetArray()} and \texttt{VecGetArrayRead()} are matched by \texttt{VecRestoreArray()} and \texttt{VecRestoreArrayRead()}.  These methods ``free-up,'' but do not de-allocate, the \pVecs, so that they can be read or written by other methods.  In general:
\begin{quote}
\emph{Each \emph{\texttt{VecGetArray()}}-type call should be matched by the corresponding \emph{\texttt{VecRestoreArray()}} call once you are done with that \emph{\pVec}}.
\end{quote}

The actual content of \texttt{FormFunction()} is to implement formulas \eqref{eq:nl:expcircleF}.  Note \texttt{PetscExpReal()} computes the exponential function $e^x$.  In fact it is just an alias for \texttt{exp()} from the standard library (i.e.~\texttt{math.h}), but use of such \PETSc library functions means that the \PETSc configuration can link to consistent libraries.  In any case, we access them just by including \texttt{petsc.h}.

It is time to run this example.  We use option \texttt{-snes\_monitor}, which counts the Newton iterations and shows the residual norm $\|\bF(\bx_k)\|_2$:
\begin{cline}
$ cd c/ch4/
$ make expcircle
...
$ ./expcircle -snes_fd -snes_monitor
  0 SNES Function norm 2.874105323289e+00 
  1 SNES Function norm 8.591393113962e-01 
  2 SNES Function norm 1.609958353862e-01 
  3 SNES Function norm 1.106891696425e-02 
  4 SNES Function norm 6.618141730691e-05 
  5 SNES Function norm 2.420782802130e-09 
Vec Object: 1 MPI processes
  type: seq
0.319632
0.947542
\end{cline}
%$
Thus after 5 iterations the Newton method has reduced the residual norm by a factor of $10^9$ and stopped with solution $x_0=0.319632$ and $x_1=0.947542$.  Compare Figure \ref{fig:expcirclebasic}.

The above run also uses option \texttt{-snes\_fd}, the purpose of which the reader may already see.  Clearly the Newton iteration \eqref{eq:nl:newtoneq} requires the Jacobian, but we have only supplied the \pSNES with an implementation of function $\bF(\bx)$, not with $J_{\bF}(\bx)$.  The entries of the latter matrix are derivatives, however, and we can approximate these by finite differences.  An entry in matrix $J=J_{\bF}(\bx)$ is approximated
\begin{equation}
J_{ij} = \frac{\partial F_i}{\partial x_j} \approx \frac{F_i(\bx+\delta \be_j) - F_i(\bx)}{\delta}  \label{eq:nl:examplefdjac}
\end{equation}
if $\delta>0$ and $\be_j\in \RR^N$ denotes a vector with entry one in the $j$th position and zero otheriwse.  It turns out that choosing $\delta = \sqrt{\eps}$, where $\eps$ is machine precision, or a small amount larger than that, gives a reasonably accurate approximation if the inputs to $\bF$ are all of order approximately one \citep{Kelley2003}.

Thus with option \texttt{-snes\_fd}, the following is what is done by \pSNES internally, at least in outline, to implement the Newton method \eqref{eq:nl:newtoneq} and \eqref{eq:nl:newtonupdate} on a system of dimension $N$:
\renewcommand{\labelenumi}{(\emph{\roman{enumi}})}
\begin{enumerate}
\item from the current iterate $\bx_k$, $\bF(\bx_k)$ is evaluated using the call-back function set in \texttt{SNESSetFunction()}, i.e.~\texttt{FormFunction()},
\item then \texttt{FormFunction()} is called $N$ more times to evaluate $\bF(\bx_k+\delta \be_j)$, for $j=0,\dots,N-1$,
\item all entries of matrix $J_{\bF}(\bx_k)$ are approximated using \eqref{eq:nl:examplefdjac},
\item $N\times N$ linear system \eqref{eq:nl:newtoneq} is solved,
\item vector update \eqref{eq:nl:newtonupdate} is done,
\item a convergence test is made, and we repeat at (\emph{i}) if not converged.
\end{enumerate}

In (\emph{i}) and (\emph{ii}) together we need to evaluate \texttt{FormFunction()} $N+1$ times per Newton iteration.  While this is no particular problem for a two-dimensional example, as here, it is a worrying amount of work to do if $N$ is large, as it would be for a system of nonlinear equations coming from discretizing a PDE, or  if $\bF$ is an expensive function to evaluate.  For some PDEs, a graph-theoretic ``coloring'' algorithm makes this stage efficient; see the example at the end of this Chapter, and the example in the next Chapter.

If you run without option \texttt{-snes\_fd} then you get an error message about an un-assembled matrix:
\begin{cline}
$ ./expcircle
[0]PETSC ERROR: --------------------- Error Message -------------------------
[0]PETSC ERROR: Object is in wrong state
[0]PETSC ERROR: Matrix must be assembled by calls to MatAssemblyBegin/End();
...
\end{cline}
%$
This message is somewhat opaque unless you are conscious of the need to form the Jacobian matrix at each Newton iteration.


\section{Residual norm in the Newton iteration}

Option \texttt{-snes\_rtol} specifies by what factor the \pSNES should try to reduce the residual norm.  The default accuracy corresponds to \texttt{-snes\_rtol 1.0e-8}; the default can be seen by running
\begin{cline}
$ ./expcircle -snes_fd -help | grep snes_rtol
\end{cline}
%$
The example
\begin{cline}
$ ./expcircle -snes_fd -snes_monitor -snes_rtol 1.0e-14
\end{cline}
%$
thus asks for much more accuracy than the earlier run which used the default.  It may be a surprise that this tight-tolerance run, asking for a further $10^6$ reduction in residual norm, requires only one more iteration, namely six iterations this time, but this is typical of the Newton iteration in the best cases.  Instead of showing the Newton iterations as text output, Figure \ref{fig:newtonconvbasic} shows these residual norm values in a graph with log-scaling on the $y$-axis.

\begin{figure}
\includegraphics[width=0.8\textwidth]{newtonconvbasic}
\caption{The characteristic look of the quadratic convergence of the Newton iteration: the residual norm drops abruptly.}
\label{fig:newtonconvbasic}
\end{figure}

The residual drops very abruptly in the Figure, reflecting the hoped-for best-case behavior of Newton iteration.  The residual norm, and the error in the solution also, decreases substantially at each iteration in the sense that the error is proportional to the \emph{square} of the error at the last iteration.  The next theorem expresses such best-case behavior \citep[Theorems 1.1 and inequalities (1.13)]{Kelley2003}:

\begin{theorem}
Suppose that $\bF:\RR^N\to\RR^N$ is differentiable, $\bx^*$ is a solution of \eqref{eq:nl:equation}, $J_{\bF}$ is Lipschitz near $\bx^*$, and $J_{\bF}(\bx^*)$ is a nonsingular matrix.  Let $\be_k=\bx_k-\bx^*$, let $\|\cdot\|$ denote a vector norm and its induced matrix norm, and let $\kappa(A)=\|A^{-1}\| \|A\|$ denote the condition number of an invertible matrix $A$.  If $\bx_0$ is sufficiently close to $\bx^*$ then, in exact arithmetic,
\renewcommand{\labelenumi}{(\roman{enumi})}
\begin{enumerate}
\item there is $K\ge 0$ such that for all $k$ sufficiently large,
\begin{equation}
	\|\be_{k+1}\| \le K \|\be_k\|^2, \label{eq:nl:quadraticconvergence}
\end{equation}
\item and if $\kappa = \kappa\left(J_{\bF}(\bx^*)\right)$, so that $\kappa\ge 1$, then
	$$\frac{\|\be_k\|}{4 \kappa \|\be_0\|} \le \frac{\|\bF(\bx_k)\|}{\|\bF(\bx_0)\|} \le \frac{4 \kappa \|\be_k\|}{\|\be_0\|}.$$
\end{enumerate}
\end{theorem}

By definition, a sequence $\{\bx_k\}$ in $\RR^N$ \emph{converges quadratically to} $\bx^*$ if the sequence $\be_k=\bx_k-\bx^*$ satisfies \eqref{eq:nl:quadraticconvergence} for some $K\ge 0$.  Thus the first idea in the Theorem is that, under strong assumptions about the regularity and nonsingularity of the Jacobian, the error decays very rapidly in the sense that the iterates converge quadratically to a solution of \eqref{eq:nl:equation}.  This remarkable fact says Newton iteration is a powerful tool.  Heuristically, once the error norm $\|\be_k\|$ is a small number (e.g.~$\|\be_k\| \ll 1$), the number of correct digits in $\bx_k$ \emph{doubles} with each additional iteration.

We seem to see quadratic converge in Figure \ref{fig:newtonconvbasic}, but this Figure shows the residual norm $\|\bF(\bx_k)\|_2$ and not the error norm $\|\be_k\|_2$.  The second part of the Theorem says that the relative error decrease at the $k$th iteration (i.e.~$\|\be_k\|/\|\be_0\|$) is within a factor, determined by the conditioning of the Jacobian at the solution, of the relative residual norm decrease at the $k$th iteration (i.e.~$\|\bF(\bx_k)\|/\|\bF(\bx_0)\|$).  This idea is especially significant because the latter quantity, the residual norm reduction ratio, is \emph{computable}.

The Theorem therefore confirms that residual norm decay like that shown in Figure \ref{fig:newtonconvbasic} corresponds to quadratic convergence of $\bx_k$ to a solution $\bx^*$.  If we want to reduce the (generally-unknowable) numerical error $\be_k=\bx_k-\bx^*$ by a given amount then it suffices to reduce the residual norm by a comparable amount.  The factor $4 \kappa$ by which the two relative norms differ is large only if the conditioning of the Jacobian at the solution is poor.  A large condition number $\kappa=\kappa\left(J_{\bF}(\bx^*)\right)$ would also mean, just as in the linear case, that there is lost precision in solving the nonlinear equations by \emph{any} numerical means; recall the numerical facts-of-life from Chapter \ref{chap:linearsystem}.

Residual norm reduction is exactly what the option \texttt{-snes\_rtol} controls, that is, the iteration continues until
    $$\frac{\|\bF(\bx_k)\|_2}{\|\bF(\bx_0)\|_2} \le \text{\texttt{snes\_rtol}}.$$
Actually, to give the more complete story, there are \emph{three} \pSNES tolerances, listed in Table \ref{tab:snestolerances}.  The iteration stops as soon as one of these conditions is satisfied.  Note that $\bs_k$ denotes the solution to linear system \eqref{eq:nl:newtoneq}, the ``step'' at iteration $k$.  The defaults for the three tolerances are \texttt{X}$=10^{-8},10^{-50},10^{-8}$, respectively.

\begin{table}
\begin{tabular}{lll}
\underline{Option}\hspace{0.2in} & \underline{Name}\hspace{0.2in} & \underline{Condition}\hspace{0.2in} \\
\texttt{-snes\_rtol X} & relative (\texttt{FNORM\_RELATIVE}) & $\|\bF(\bx_k)\|_2 \le \text{\texttt{X}}\, \|\bF(\bx_0)\|_2$ \\
\texttt{-snes\_atol X} & absolute (\texttt{FNORM\_ABS}) & $\|\bF(\bx_k)\|_2 \le \text{\texttt{X}}$ \\
\texttt{-snes\_stol X} & step-length (\texttt{SNORM\_RELATIVE}) & $\|\bs_k\|_2 \le \text{\texttt{X}}\, \|\bx_k\|_2$
\end{tabular}
\caption{The three ways \pSNES can succeed, i.e.~the tolerance options and corresponding termination conditions, for stopping the Newton iteration.} \label{tab:snestolerances}
\end{table}

\medskip
Option \texttt{-snes\_converged\_reason} reports which termination condition was active, using the parenthetical name given in Table \ref{tab:snestolerances}.  For example,
\begin{cline}
$ ./expcircle -snes_fd -snes_converged_reason
Nonlinear solve converged due to CONVERGED_FNORM_RELATIVE iterations 5
...
\end{cline}
%$

So far we have portrayed the Newton iteration in optimistic terms, but it is not magic and things can go wrong.  First note a key hypothesis in the above Theorem, namely that ``$\bx_0$ is sufficiently close to $\bx^*$''.  Even on well-behaved nonlinear equations, if $\bx_0$ is far from the solution then the iteration may take many steps before $\|\be_k\|$ becomes small enough so that quadratic convergence \eqref{eq:nl:quadraticconvergence} ``kicks in''.  For example, Figure \ref{fig:newtonconvdelayed} shows what happens if we use initial iterate $\bx_0=[10\,\, 10]^\top$ in the above Example.  A long period of slower convergence\sidenote{Evidently \emph{linear} convergence in which the residual norm is being reduced by a constant factor.} is needed before the iterate enters the region where $\bx_0$ is sufficiently close so that the conclusions of the Theorem apply.  This region is sometimes known as the ``ball of quadratic convergence.''

\begin{figure}
\includegraphics[width=0.8\textwidth]{newtonconvdelayed}
\caption{Even in for well-behaved systems $\bF(\bx)=0$, if the initial iterate $\bx_0$ is far from the solution then quadratic convergence can be postponed for many iterations.}
\label{fig:newtonconvdelayed}
\end{figure}

On other equations the Newton iteration \eqref{eq:nl:newtoneq}, \eqref{eq:nl:newtonupdate} as it stands actually diverges.  In those cases the ``linesearch'' schemes in \PETSc will reduce the step length $\|\bs_k\|$.  These schemes, which are addressed later in this section, are effective at ``globalizing'' the convergence behavior \citep{Kelley2003}.  For an example, see Exercise \ref{chap:nonlinear}.\ref{exer:newtonatan}.

Finally, many practical problems do not have the smoothness needed to apply the above Theorem.  In such cases the problem may need regularization, continuation, or other procedures to make it manageable.


\section{Exact Jacobians and passing parameters}

We have yet to exploit two critical parts of the \pSNES API, namely the ability to provide an exact Jacobian, which is to say a method which computes the Jacobian function $J_{\bF}(\bx)$, and the ability to pass parameters through the call-back mechanism so that such parameters can be used inside the residual- and Jacobian-evaluation functions.  The next code \texttt{expcircleJAC.c}, in Figures \ref{code:expcircleJACI} and \ref{code:expcircleJACII}, uses these abilities.  For later reference, this code is proposed as a ``model use'' of \pSNES.

The first new idea seen in Figure \ref{code:expcircleJACI} is the declaration of a C \texttt{struct} called \texttt{AppCtx} (``application context'').  It has just one element, the real parameter $b$ which appears in formulas \eqref{eq:nl:expcircleF} and \eqref{eq:nl:expcircleJac}.  A \texttt{struct} is not really necessary here, but it will be needed in future examples where there is more than one parameter to pass.

\cinputpart{expcircleJAC.c}{\CODELOC}{Solves the same nonlinear system as does \texttt{expcircle.c}, but this version includes an exact Jacobian and has the \pSNES pass a parameter into the call-back functions.}{I}{//START}{//END}{code:expcircleJACI}

Next, the method \texttt{FormFunction()} in Figure \ref{code:expcircleJACI} is almost the same as the one in \texttt{expcircle.c}, namely in Figure \ref{code:expcircle}, but the value of $b$ comes from the \texttt{struct} instead of being hard-wired as before.  In detail, the argument \texttt{void *ctx} is ``cast'' in the sense of the C language \citep{KernighanRitchie1988} to a pointer of type \texttt{AppCtx*}, and then the parameter is extracted via the pointer (i.e.~by ``\texttt{user->b}'' which is shorthand for ``\texttt{(*user).b}'').  This awkward-seeming method of passing parameters allows the signature of \texttt{FormFunction()} to be precisely as before.

\cinputpart{expcircleJAC.c}{\CODELOC}{The \texttt{main()} method is also similar to that in \texttt{expcircle.c}, but with a bit more code to allocate the \pMat which stores the Jacobian.}{II}{//STARTMAIN}{//ENDMAIN}{code:expcircleJACII}

The method \texttt{FormJacobian()} in Figure \ref{code:expcircleJACI} is new.  It has similar structure and semantics to \texttt{FormFunction()}.  Like that method, there is a required signature so that it can be used in call-back, namely
\begin{code}
PetscErrorCode (*J)(SNES,Vec,Mat,Mat,void*)
\end{code}
Input \pVec \texttt{x} and parameter-passing pointer \texttt{void *ctx} have the same meaning as in \texttt{FormFunction()}.  For reading \texttt{x} we use \texttt{VecGetArrayRead()} and \texttt{VecRestoreArrayRead()} just as in the previous code \texttt{expcircle.c}.

The difference is that we must set a \pMat as output, based on formula \eqref{eq:nl:expcircleJac} in this case, not a \pVec like \texttt{F}.  To set values in the output \pMat we use \texttt{MatSetValues()}, so the roles of real array \texttt{v[4]}, as values, and integer arrays \texttt{row[2],col[2]}, as global indices, should be familiar from Chapter \ref{chap:linearsystem}.

An interesting detail now appears!   It will take some explanation.  There are actually \emph{two} output \pMats for \texttt{FormJacobian()} to set.  The first, called \texttt{J} here, corresponds to the Jacobian matrix itself, which, in this simple case, we want to supply.  The second, \texttt{Jpre}, might be a very poor approximation of the Jacobian; it is the matrix we would supply as material to build a preconditioner.

Now is the time to look at Figure \ref{code:expcircleJACII}, and note the differences from \texttt{main()} in the earlier code \texttt{expcircle.c} (Figure \ref{code:expcircle}).  There are two differences, the first of which is obvious and unrelated to the Jacobian; namely, we declare and initialize a \texttt{struct} instance to pass the parameter $b$.

The second difference is that we create and configure a $2\times 2$ \pMat \texttt{J} to hold the Jacobian.  Our use of \texttt{MatCreate(), MatSetSizes(), MatSetFromOptions(),} and \texttt{MatSetUp()} on \texttt{J} mimics what we did for finite-dimensional systems in Chapter \ref{chap:linearsystem}.  However, this time when we set-up the \pSNES we pass \texttt{J} for two arguments,
\begin{code}
SNESSetJacobian(snes,J,J,FormJacobian,&user);
\end{code}
This means we give a pointer to the allocated space \pMat \texttt{J} as both the Jacobian matrix and as the preconditioner matrix.  FIXME


\section{``Matrix-free'' Newton method}

FIXME: theory of matrix-free JK \cite{KnollKeyes2004}

FIXME because of double-mat-assembly in \texttt{FormJacobian()}, all of these work

\begin{cline}
$ ./expcircleJAC -snes_monitor -snes_mf_operator
$ ./expcircleJAC -snes_monitor -snes_mf
$ ./expcircleJAC -snes_monitor -snes_fd
$ ./expcircleJAC -snes_monitor
\end{cline}


\section{Line search}

There remain two major ideas not covered above, i.e.~beyond the construction of the Newton iteration \eqref{eq:nl:newtoneq} itself, to turn Newton iteration into an effective tool:
\renewcommand{\labelenumi}{\roman{enumi})}
\begin{enumerate}
\item linesearch or trust region needed \citep{Kelley2003}
\item full range of linear tools (e.g.~Chapter \ref{chap:linearsystem}) should be applied to the linear system
\end{enumerate}

Regarding the former FIXME


\section{Example: 1D reaction-diffusion equation}

FIXME

\vfill
\cinputpart{reaction.c}{\CODELOC}{FIXME}{I}{//SETUP}{//ENDSETUP}{code:reactionI}

\cinputpart{reaction.c}{\CODELOC}{FIXME}{II}{//FUNJAC}{//ENDFUNJAC}{code:reactionII}

\cinputpart{reaction.c}{\CODELOC}{FIXME}{III}{//MAIN}{//ENDMAIN}{code:reactionIII}


\section{Exercises}

\renewcommand{\labelenumi}{\arabic{chapter}.\arabic{enumi}\quad}
\renewcommand{\labelenumii}{(\alph{enumii})}
\begin{enumerate}
\item One needs to \emph{see} quadratic convergence to believe it.  Observe that for $x_k$ in both parts (b) and (c), the \emph{number of correct digits in $x_k$ doubles at each iteration}.
    \begin{enumerate}
    \item The sequence $x_k = 1-2^{-n}$ converges quickly to $x^*=1$, but not quadratically.  Find $k$ so that $|e_k| < 10^{-16}$.
    \item The sequence $x_k = 1-2^{-2^n}$ converges quadratically to $x^*=1$.  Find $k$ so that $|e_k| < 10^{-16}$.  Find the smallest $K$ so that $|e_{k+1}| \le K |e_k|^2$ for all $k$.
    \item Let $F(x) = \cos(x-1) - \exp(1-x)$ and $x_0=0.5$.  Using any quick-and-dirty numerical tool,\sidenote{Extra credit for doing it in \PETSc.} compute Newton iterates $x_k$ for $k=1,\dots,6$.  Estimate $K$ so that $|e_{k+1}| \le K |e_k|^2$ for large $k$.
    \end{enumerate}

\item Make a tiny modification to \texttt{expcircleJAC.c} to set the initial vector to $\bx_0 = [10\,\, 10]^\top$.  Rerun it with option \texttt{-snes\_monitor} and note it does not converge to a solution.  Why does the \pSNES stop?  Add one runtime option so that it does converge to a solution, thereby producing the data shown in Figure \ref{fig:newtonconvdelayed}.

\item  \label{exer:newtonatan}
This example shows a famous case where the no-linesearch Newton iteration either diverges or enters a limit cycle according to the initial iterate $x_0$.  Linesearch fixes this.
    \begin{enumerate}
    \item By simple modifications of \texttt{expcircle.c}, write a code \texttt{e3atan.c} which solves the (one-dimensional) nonlinear equation $F(x)=\arctan(x)=0$ using initial iterate $x_0=2.0$.  Use \texttt{atan()}.  By running it \texttt{./e3atan -snes\_fd -snes\_linesearch\_type basic}, show that the Newton iteration without a linesearch diverges, while using the default choice \texttt{-snes\_linesearch\_type bt} gives convergence.
    \item Continuing with $F(x)=\arctan x$, require the Newton iteration \eqref{eq:nl:newtoneq}, \eqref{eq:nl:newtonupdate} to enter a limit cycle, i.e.~so that $x_{k+1} = - x_k$ for all $k$.  Thereby approximate the positive value $x_0$ so that the Newton iteration makes no progress, and yet remains bounded, for all $k$.\sidenote{One may solve this problem with Newton's method.}  Draw a (well-known) sketch of the situation.  Then, by modifying $x_0$ in \texttt{e3atan.c} and again using linesearch type \texttt{basic}, confirm that one can make the Newton iteration ``stick'' in this (actually unstable) limit cycle for quite a while.  Then confirm that linesearch types \texttt{bt,l2,cp} all get unstuck immediately.
% to solve:  - x = x - arctan(x) (1+x^2)
% octave:
%   G = @(x) 2*x - atan(x) * (1+x^2);
%   J = @(x) 1 - atan(x) * (2*x);
%   x = 1.4;
%   format long g
%   for j = 1:4, x = x - G(x) / J(x), end
% get:
%  x = 1.39174520027073
    \end{enumerate}

\end{enumerate}
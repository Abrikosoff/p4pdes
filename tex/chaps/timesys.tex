
To the reader who has become lost in a maze of issues which have built up around the examples so far---Krylov solvers, preconditioner choices, line searches, finite element details, or whatever causes confusion---some good news.  In \emph{this} Chapter we restart with an easy question and a small example.

Solving ordinary differential equations (ODEs) with \PETSc ought to be easy.  Given that we have practice with \PETSc objects, it is.  Furthermore, the transition to numerical methods for time-dependent, parabolic PDEs like the heat equation is straightforward.

In this Chapter we look at three examples:
\begin{itemize}
\item a linear system of two ODEs,
\item an arbitrarily-large system of ODEs arising from spatial finite-difference approximation of the 2D time-dependent heat PDE,
\item and another large ODE system arising from two coupled nonlinear PDEs in 2D, forming a two-species reaction/diffusion model.
\end{itemize}
These problems are solved using a new \PETSc object, the \pTS time-stepping solver.

The particular integration method used by the \pTS can be chosen at run-time.  We will recall the basics of both \emph{explicit} and \emph{implicit} methods.  Because implicit time-stepping implies the full stack of \pSNES/\pKSP/\pPC solvers for the (generally) non-linear equations at each time step, our experience with \pSNES examples, from previous Chapters, will come in handy.

User code will discretize the two PDE examples in space only.  This \emph{method of lines} approach, which is the most common way to solve PDEs using \PETSc \pTS, avoids any need to write code specific to a temporal discretization.

The third example is the first \emph{system} of PDEs addressed so far.  While it requires some new constructs, implementing this system in \PETSc code is straightforward.  However, our next systems example, the Stokes equations in Chapter \ref{chap:ok}, will have more interesting implementation details and solver choices.

\section{Systems of ODEs}  Consider an ODE system in form
\begin{equation}
\by' = \bg(t,\by)  \label{eq:ts:ode}
\end{equation}
where $\by(t) \in \RR^N$, $\by'=d\by/dt$, and $g(t,\by)$ is continuous in both inputs.  Suppose an \emph{initial value} is given:
\begin{equation}
\by(t_0) = \by_0.  \label{eq:ts:ode:iv}
\end{equation}
Under additional, but reasonable, assumptions on the behavior of $\bg$, the continuum problem \eqref{eq:ts:ode}, \eqref{eq:ts:ode:iv} is a well-posed problem, at least for short times; this much is well-known.  Thus one can make predictions from such a ODE \emph{initial value problem} (IVP), forward or backward from the initial time.

Let us be specific about sufficient regularity of $\bg$ to imply well-posedness.  We will assume for simplicity that $\bg(t,\by)$ is continuous on a cylinder around the initial point $(t_0,\by_0)$,
   $$\mathcal{D} = \{(t,\by) \,:\, |t-t_0| \le \delta, \|\by - \by_0\| \le \omega\}$$
where $\delta > 0$ and $\omega > 0$.  Then we further assume that $\bg$ is \emph{Lipschitz} in its second argument, so that output differences are bounded by a multiple of input differences: there is $L\ge 0$ so that
\begin{equation}
\|\bg(t,\by_1) - \bg(t,\by_0)\| \le L \|\by_1-\by_0\|  \label{eq:ts:glipschitz}
\end{equation}
for $(t,\by_i) \in \mathcal{D}$.  Then the problem \eqref{eq:ts:ode}, \eqref{eq:ts:ode:iv} has a unique continuous, and continuously-differentiable, solution $\by(t)$ on a generally-shorter interval, namely $|t-t_0|<\eps$ for some $0 < \eps \le \delta$ \citep[section 17.5]{HirschSmaleDevaney2004}.  Furthermore the solution depends continuously on both the initial value $\by_0$ and the right-hand-side $\bg$.

Our brief attention to well-posedness is motivated by the following practical concern:  When we run a numerical differential equation code it will produce numbers.\sidenote{It \emph{always} produces numbers!}  These numbers are (essentially) never the exact solution of the differential equation, but they are ``correct'' in a numerical-analysis sense if we can demonstrate convergence to the solution of a well-posed continuum problem.  Thus we accept ``wrong'' numbers as correct in cases where demonstrably convergent numerical analysis is applied to a well-posed problem.

\emph{Much} more egregiously wrong, however, are numbers which represent no continuum solution at all.  Benign-looking scalar nonlinear ODEs can put us in such peril.  In fact, Exercise \ref{chap:ts}.\ref{exer:ts:tan} gives a simple, and well-known, example where the solution ceases to exist after a certain finite interval of time.  The approximating code sails right by the end of this interval of time, producing numbers that are infinitely-erroneous.  The examples in this book include several nonlinear PDEs, where this concern is yet greater.  Thus we will be cautious.  When possible we consider well-posedness, and perhaps even approximate-ability (regularity), prior to implementation.

For linear systems, solutions exist for all time.  The following example ODE system is thus a safe starting point.

\noindent\hrulefill
\begin{example}  \label{ex:ts:odeeasy} Consider the initial value problem in two dimensions
\begin{equation}
   \by' = \begin{bmatrix} y_1 \\ - y_0 + t\end{bmatrix}, \qquad \by(0) = \begin{bmatrix} 0 \\ 0 \end{bmatrix}, \label{eq:ts:example}
\end{equation}
from initial time $t_0=0$.  This is a linear system with
    $$\bg(t,\by) = A \by + \bbf \,\text{ where } A = \begin{bmatrix} 0 & 1 \\ -1 & 0 \end{bmatrix} \text{ and } \bbf(t) = \begin{bmatrix} 0 \\ t\end{bmatrix}.$$
The exact solution, shown in Figure \ref{fig:ts:ode}, is
\begin{equation}
    \by(t) = \begin{bmatrix} t - \sin t \\ 1 - \cos t \end{bmatrix}.\label{eq:ts:examplesoln}
\end{equation}

\vspace{-3mm}
\begin{figure}
\includegraphics[width=0.7\textwidth]{figs/ode}
\caption{The solution to \eqref{eq:ts:example} is given by \eqref{eq:ts:examplesoln}, with $y_0(t)$ solid and $y_1(t)$ dashed.}
\label{fig:ts:ode}
\end{figure}
\end{example}
\vspace{-5mm}
\noindent\hrulefill

\section{Methods, and their accuracy, for ODE initial value problems}  The above example ODE system is straightforward to approximate by any number of time-stepping numerical methods.  They all start from the initial values and use the right-hand-side function $g(t,\by)$ to update an approximate solution.  They all generate sequences $\{\bY_\ell\}$ which approximate the solution at discrete times $t_\ell$, so that $\bY_\ell \approx \by(t_\ell)$.  Because these ideas are easy to find in the literature\sidenote{A good reference is \citep{AscherPetzold1998}, for example.}, and are likely to be in the reader's background in some form, we only supply a gloss here.

Let $t_0<t_1<t_2<\dots<t_L$ be a finite sequence of times, with steps $\Delta t_\ell = t_\ell - t_{\ell-1}$ for $\ell=1,2,\dots,L$.  Let $\bY_0 = \by_0$ be the known starting value  \eqref{eq:ts:ode:iv}.  \emph{Euler's method}, also known as the forward Euler method, generates $\bY_1,\bY_2,\dots,\bY_L$ from
\begin{equation}
\frac{\bY_\ell - \bY_{\ell-1}}{\Delta t_\ell} = \bg(t_{\ell-1},\bY_{\ell-1}). \label{eq:ts:forwardeuler}
\end{equation}
One can also write this as a formula for the updated value $\bY_\ell$,
\begin{equation}
\bY_\ell = \bY_{\ell-1} + \Delta t_\ell \, \bg(t_{\ell-1},\bY_{\ell-1}). \label{eq:ts:forwardeulerupdate}
\end{equation}

The \emph{backward Euler method} is just as easy to state,
\begin{equation}
\frac{\bY_\ell - \bY_{\ell-1}}{\Delta t_\ell} = \bg(t_{\ell},\bY_{\ell}),  \label{eq:ts:backwardeuler}
\end{equation}
but much more work is required to use it in practice.  Indeed, just writing the analog of \eqref{eq:ts:forwardeulerupdate}, namely a formula by which to compute $\bY_\ell$, would require doing algebra on the function $\bg$.  Backward Euler is an \emph{implicit} method, as explored below.  At each time step \eqref{eq:ts:backwardeuler} generates exactly the kind of (generally) nonlinear system, with unknown values $\bY_\ell$, which we solved in previous Chapters by Newton's method and \pSNES.

Both methods \eqref{eq:ts:forwardeuler} and \eqref{eq:ts:backwardeuler}, which are the simplest approximations of \eqref{eq:ts:ode}, compute a first-order finite difference approximation of the derivative of $\by(t)$ by evaluating the right-hand side function $\bg(t,\by)$ at a particular point.  They differ only in the evaluation point.  They are ``first-order accurate'' in the simple sense that, for any twice-continuously-differentiable function $f(t)$, by Taylor's theorem
    $$\frac{f(t+\Delta t) - f(t)}{\Delta t} = f'(t) + O(\Delta t)$$
as $\Delta t \to 0$.  One says that the Euler methods have first-order \emph{local truncation error}.

However, under sufficient assumptions of smoothness, the deeper meaning of ``first-order accuracy'' also applies, as a rate of convergence as $\Delta t\to 0$.  One would hope that the approximations $\bY_\ell$ converge to the values $\by(t_\ell)$ in the same limit, but notice this requires knowing how the errors build up over many steps, and not just knowing the local truncation error at each step.

Regarding what is provable, by assuming that the solution $\by(t)$ exists, that a Lipschitz bound \eqref{eq:ts:glipschitz} applies to $\bg(t,\by)$, and that the solution has bounded second derivative ($|\by(t)''|\le M$), all on some interval $[t_0,t_f]$, one can show that the Euler methods satisfy $|\bY_\ell-\by(t_\ell)| = O(\Delta t^1)$.  In slightly more detail, one can show \citep{AscherPetzold1998}
\begin{equation}
|\bY_\ell-\by(t_\ell)| \le \frac{\Delta t\,M}{2L} (e^{L(t_\ell-t_0)}-1).  \label{eq:ts:eulerbound}
\end{equation}

The \emph{numerical error} $E_\ell = |\bY_\ell-\by(t_\ell)|$, also called the \emph{global truncation error}, is first-order in $\Delta t$ for the Euler methods: $E_\ell = O(\Delta t^1)$.  This statement is the intended meaning of ``first-order accuracy'' for these methods, while bound \eqref{eq:ts:eulerbound} itself is rarely used quantitatively.\sidenote{Not least because we apply higher-order methods, as below.}  The order of some method, stated as $E_\ell = O(\Delta t^p)$ for some $p>0$, tells us what to expect when we shorten the time step, namely by how much a reduction of the time step will reduce the numerical error.  Halving the time step in a first-order method should reduce the numerical error by half, for a second-order method by a fourth, and so on.

\newcommand{\RKtwoa}{RK$2$a\xspace}
\newcommand{\RKthreebs}{RK$3$bs\xspace}
\newcommand{\RKfour}{RK$4$\xspace}

It should be no surprise that there are methods with improved accuracy compared to the Euler methods.  Among these are the \emph{one-step multi-stage} methods, with the \emph{Runge-Kutta} family best known.  One-step methods use $\bY_{\ell-1}$ only, and not previous values, to build the next approximation $\bY_\ell$, but generally by multiple evaluations of the right-hand side $\bg$.  For example, denoting the time-step simply as $h=\Delta t_\ell$, the \emph{explicit trapezoidal} method \citep{AscherPetzold1998} is a second-order rule which takes a forward Euler step but then ``goes back'' and recomputes the step by the average of values of $\bg$:
\begin{align}
\hat\bY &= \bY_{\ell-1} + h\, \bg(t_{\ell-1},\bY_{\ell-1}) \label{eq:ts:rk2a} \\
\bY_\ell &= \bY_{\ell-1} + \frac{h}{2} \bg(t_{\ell-1},\bY_{\ell-1}) + \frac{h}{2} \bg(t_\ell,\hat\bY). \notag
\end{align}
Though two evaluations of $\bg$ (i.e.~the two \emph{stages}) occur at each time step, only the final result $\bY_\ell$ is used to inform the next time step (i.e.~the one which computes $\bY_{\ell+1}$).  In this sense the method is ``one-step'' despite being multi-stage.  Method \eqref{eq:ts:rk2a} is also called ``\RKtwoa;'' it is one of several possible second-order Runge-Kutta methods.

The well-known fourth-order Runge-Kutta method ``\RKfour'' has four stages.  Instead of stating this particular scheme using formulas like those in \eqref{eq:ts:rk2a} above, let us define tabular notation for \emph{any} one-step, $s$-stage method.  Namely, the $s+1$ formulas
\begin{align}
\hat\bY_i &= \bY_{\ell-1} + h \sum_{j=1}^s a_{ij}\, \bg(t_{\ell-1} + c_j h, \hat\bY_j), \qquad 1 \le i \le s \label{eq:ts:rkgeneral} \\
\bY_\ell  &= \bY_{\ell-1} + h \sum_{i=1}^s b_i\, \bg(t_{\ell-1} + c_i h, \hat\bY_i) \notag
\end{align}
correspond to the \emph{tableau} \citep{Butcher2008}
\begin{center}
\begin{tabular}{c|cccc}
$c_1$    & $a_{11}$ & $a_{12}$ & $\cdots$ & $a_{1s}$ \\
$c_2$    & $a_{21}$ & $a_{22}$ & $\cdots$ & $a_{2s}$ \\
$\vdots$ & $\vdots$ & $\vdots$ & $\ddots$ & $\vdots$ \\
$c_s$    & $a_{s1}$ & $a_{s2}$ & $\cdots$ & $a_{ss}$ \\ \hline
         & $b_1$    & $b_2$    & $\cdots$ & $b_s$
\end{tabular}
\end{center}

For example, the $s=2$ method \RKtwoa is the left tableau in Table \ref{tab:ts:exampletableau}, and the $s=4$ method \RKfour is the middle one.  Spaces in the tableau correspond to zeros.  Note that, in all three methods in Table \ref{tab:ts:exampletableau}, $\hat\bY_1=\bY_{\ell-1}$ so the first rows are trivial.  Also, because these schemes are explicit they have nonzero $a_{ij}$ only strictly below the diagonal; we return to this idea below.

\begin{table}
\begin{tabular}{c|cc}
$0$   \\
$1$ & $1$ \\ \hline
    & $\frac{1}{2}$ & $\frac{1}{2}$
\end{tabular}
\hfill
\begin{tabular}{c|cccc}
$0$   \\
$\frac{1}{2}$ & $\frac{1}{2}$ \\
$\frac{1}{2}$ & $0$           & $\frac{1}{2}$ \\
$1$           & $0$           & $0$           & $1$  \\ \hline
              & $\frac{1}{6}$ & $\frac{1}{3}$ & $\frac{1}{3}$ & $\frac{1}{6}$
\end{tabular}
\hfill
\begin{tabular}{c|cccc}
$0$   \\
$\frac{1}{2}$ & $\frac{1}{2}$ \\
$\frac{3}{4}$ & $0$           & $\frac{3}{4}$ \\
$1$           & $\frac{2}{9}$ & $\frac{1}{3}$ & $\frac{4}{9}$ \\ \hline
              & $\frac{2}{9}$ & $\frac{1}{3}$ & $\frac{4}{9}$ & $0$ \\
              & $\frac{7}{24}$ & $\frac{1}{4}$ & $\frac{1}{3}$ & $\frac{1}{8}$
\end{tabular}
\medskip
\caption{Tableau for the explicit trapezoidal rule (\RKtwoa; left), the classical fourth-order Runge-Kutta method (\RKfour; middle), and an embedded four-stage, third-order scheme (\RKthreebs, the \PETSc default; right).} \label{tab:ts:exampletableau}
\end{table}

In summary, a one-step, multi-stage method, whether described by \eqref{eq:ts:rkgeneral} or a tableau, computes a new value $\bY_\ell$ via intermediate results $\hat\bY_i$.  For a more interesting example of this approach, we describe the default Runge-Kutta scheme in \PETSc, an \emph{adaptive} scheme which is \emph{embedded}.  Let us give meaning to these terms.

An ODE numerical solution method is \emph{adaptive} if it adjusts the time step by a procedure which uses information from the already-computed values.  An adaptive scheme will take shorter steps when the values from $\bg$ are rapidly-changing or irregular, and longer steps when they are smoother and more predictable.  Successfully implementing such strategies, the error estimation and control \citep{AscherPetzold1998} done internally by the \PETSc time-steppers (below), and not visible in our user code,\sidenote{Thank goodness!} is nontrivial and outside our scope.

One strategy to build an adaptive Runge-Kutta method is to make it \emph{embedded}.  Such a scheme is in fact two methods which have the same $c_j$ and $a_{ij}$ coefficients in their tableau, but different $b_j$ \emph{and} different orders of accuracy.  The Bogacki-Shampine scheme \RKthreebs is such an embedded pair.\sidenote{The reader may have seen it as the \texttt{ode23} method in \Matlab.}  Each is four stage, but one is of third order and one of second order.  Their tableau are shown at right in Table \ref{tab:ts:exampletableau}, with the $b_j$ coefficients listed on different rows below the horizontal line.  The different results from these two schemes is used to decide if the time step was short enough to satisfy the error tolerance/goal \citep{Butcher2008}.  In \RKthreebs specifically, one scheme (first $b_j$ row) makes $O(\Delta t^3)$ local truncation errors, while the other (second $b_j$ row) makes $O(\Delta t^2)$ errors, and the difference estimates the accuracy of the lower order scheme.  The work of computing two approximations is the same cost as just one, because they share the same evaluations of $\bg$; this is the central meaning of ``embedded.''  Further details of the \RKthreebs scheme, including its stability and efficiency properties, are beyond our scope, but see \citep{BogackiShampine1989}.

There are alternative families other than one-step schemes.  For example, the \emph{second-order Adams-Bashforth} method
\begin{equation}
\bY_\ell = \bY_{\ell-1} - \frac{h}{2} \bg(t_{\ell-2},\bY_{\ell-2}) + \frac{3 h}{2} \bg(t_{\ell-1},\bY_{\ell-1})  \label{eq:ts:ab2}
\end{equation}
is a \emph{multi-step} method.  It computes a new value $\bY_\ell$ with only one additional evaluation of $\bg$ (i.e.~one stage), but using two previous values $\bY_{\ell-2},\bY_{\ell-1}$.  This works in a straightforward way only if the step-size $h=\Delta t$ is independent of $\ell$.  A goal of such methods is to achieve order greater than one with only one evaluation of $\bg$ per time step, and this is often achieved in practice.  Other schemes combine the multiple-stages and multi-step ideas such as the \emph{general linear methods} \citep{Butcher2008}.  In our quick treatment of time-stepping, we will say no more about multi-step and general linear methods, though the latter appear as possible \PETSc options in Table \ref{tab:ts:tstypes} below.

The general theme is that the cost of an ODE numerical method is proportional to the number of (distinct) evaluations of the right-hand side $\bg$ of the ODE, especially if $\bg$ has significant nonlinearities.  The additional arithmetic in implementing a particular scheme, represented for example by the number of coefficients in the tableau, is usually less important.

Clearly there are a very large number of possible methods, many of which are available from the \PETSc command line, whether one-step or multi-step, adaptive or not, and so on.   Choosing the ``best'' one for a given ODE system is not a reasonable goal.  Our goal in this Chapter is only to suggest some reasonable choices for broad classes of problems, but we invoke an essential \PETSc mantra here, namely: a rich set of command-line, run-time options is \emph{not a bad thing}.

\section{\PETSc \pTS objects, and a first example}  The example system on page \pageref{ex:ts:odeeasy} is a good starting point for our first ODE-solving program because it is of small, fixed dimension,\sidenote{Our last fixed-dimension programs were at the start of Chapter \ref{chap:nl}.  Note that \texttt{ode.c} and \texttt{odejac.c} in the current Chapter do not use a \pDMDA grid object, but \texttt{heat.c} and \texttt{pattern.c} do.} and because it has an exact solution from which we evaluate numerical error.  The program is \texttt{c/ch6/ode.c} shown in Codes \ref{code:ts:ode:main} and \ref{code:ts:ode:callbacks}.  It uses a \PETSc time-stepping object of type \pTS.

\cinputpart{ode.c}{\CODELOC}{The \texttt{main()} part of \texttt{ode.c} creates and configures a \pTS object, plus the \pVecs for approximate and exact solutions.}{I}{//MAIN}{//ENDMAIN}{code:ts:ode:main}

The first thing the code does is to create space for the approximate and exact solutions in \pVecs of fixed size $N=2$.  Then we see these commands which initialize the \pTS:
\begin{code}
  TSCreate(PETSC_COMM_WORLD,&ts);
  TSSetProblemType(ts,TS_NONLINEAR);
  TSSetRHSFunction(ts,NULL,FormRHSFunction,NULL);
\end{code}
By setting the problem type to \texttt{TS\_NONLINEAR} we are saying that the problem is in form \eqref{eq:ts:ode} with a function $\bg(t,\by)$; we ignore the fact here that our particular system is linear.  As we will see in later examples, type \texttt{TS\_NONLINEAR} allows even more general form than equation \eqref{eq:ts:ode}.  Calling \texttt{TSSetRHSFunction()} sets a call-back to our own function \texttt{FormRHSFunction()} which evaluates $\bg(t,\by)$; see Code \ref{code:ts:ode:callbacks}.

Then we set the particular \pTS solver to Runge-Kutta:
\begin{code}
  TSSetType(ts,TSRK);
\end{code}
In fact this is nothing but a default choice, which can be overriden by option \texttt{-ts\_type}, as shown below.

After that are additional commands for configuring the time-axis of the \pTS.  All of the following choices can, and often will be, overridden by command-line options:
\begin{code}
  TSSetInitialTimeStep(ts,t0,dt);
  TSSetDuration(ts,100*(int)((tf-t0)/dt),tf-t0);
  TSSetExactFinalTime(ts,TS_EXACTFINALTIME_MATCHSTEP);
  TSSetFromOptions(ts);
\end{code}
Note that \texttt{TSSetInitialTimeStep()} sets both the initial time \texttt{t0} and the (initial) time step \texttt{dt}.  If the numerical ODE solver chosen at run-time is in fact adaptive then \texttt{dt} is \emph{only} the initial time step, which is then generally modified by the solver after the first step.  Next, \texttt{TSSetDuration()} sets both the duration of the solve (\texttt{tf-t0}) and the maximum number of steps that the solver is allowed to take.  Here that is 100 times the intended number of steps, namely \texttt{(tf-t0)/dt} based on our other choices.

Because of run-time options, the values for the initial time, duration, and initial time step may have changed after the \texttt{TSSetFromOptions()} call.  Thus, before we ask for the \pTS to solve the problem by time stepping, we get the current time (\texttt{TSGetTime()}) and set the initial values from that.  The function that evaluates the exact solution, shown in Code \ref{code:ts:ode:callbacks}, also supplies the initial value here.  After \texttt{TSSolve()} comes a bit more standard code, as in previous Chapters, which computes the norm of the numerical error and then destroys (de-allocates) the objects we created.

Code \ref{code:ts:ode:callbacks} shows function \texttt{FormRHSFunction()} which evaluates $\bg(t,\by)$ for the example on page \pageref{ex:ts:odeeasy}.  The inputs are \texttt{double t} and \pVec \texttt{y}.  We read the latter into a \texttt{const} array with \texttt{VecGetArrayRead()}, while the output \pVec \texttt{g}, which our code modifies, is accessed through \texttt{VecGetArray()}.  In other words, \texttt{FormRHSFunction()} is similar to the \texttt{FormFunction()} code we wrote in Chapter \ref{chap:nl}.

\cinputpart{ode.c}{\CODELOC}{\texttt{SetFromExact()} computes the exact solution, while \texttt{FormRHSFunction()} computes the right-hand side of \eqref{eq:ts:ode}, namely $\bg(t,\by)$.}{II}{//CALLBACKS}{//ENDCALLBACKS}{code:ts:ode:callbacks}

Let us try it out.  Two options to try, when in doubt or just curious, are to \texttt{monitor} and \texttt{view} the \pTS.  First the result of \texttt{monitor}:
\begin{cline}
$ cd c/ch6/
$ make ode
$ ./ode -ts_monitor
0 TS dt 0.1 time 0.
1 TS dt 0.170141 time 0.1
2 TS dt 0.169917 time 0.270141
3 TS dt 0.171145 time 0.440058
...
86 TS dt 0.206773 time 19.6777
87 TS dt 0.11548 time 19.8845
88 TS dt 0.205616 time 20.
error at tf = 20.000 :  |y-y_exact|_inf = 0.00930352
\end{cline}
%$
It is clear that the time-stepping method is adaptive and, given that we have computed the final numerical error, that it has essentially succeeded.  Comparing Figure \ref{fig:ts:ode}, an error of magnitude $9\times 10^{-3}$ at time $t=20$ is indeed small.

What solution method was used?  Do \texttt{-ts\_view}:
\begin{cline}
$ ./ode -ts_view
TS Object: 1 MPI processes
  type: rk
  maximum steps=20000
  maximum time=20.
  total number of nonlinear solver iterations=0
  total number of nonlinear solve failures=0
  total number of linear solver iterations=0
  total number of rejected steps=12
    RK 3bs
    Abscissa     c =  0.000000  0.500000  0.750000  1.000000 
  FSAL: yes
  TSAdapt Object:   1 MPI processes
...
error at tf = 1.000 :  |y-y_exact|_inf = 0.000144484
\end{cline}
%$
As asserted earlier, the default for type \texttt{TSRK} is \RKthreebs.  Interestingly, the adaptive method tried and then rejected 12 steps in the 88 taken.

As with other major \PETSc solver types, there are many control options, so a first response is to get help:
\begin{cline}
$ ./ode -help |grep ts_
\end{cline}
%$

For example, to see numerical values of the solution at each time-step, do
\begin{cline}
$ ./ode -ts_monitor_solution
\end{cline}
%$
For a run-time graphical (line-graph) view of the solution, do
\begin{cline}
$ ./ode -ts_monitor -ts_monitor_lg_solution -draw_pause 0.1
\end{cline}
%$
The result is a movie version of Figure \ref{fig:ts:ode}.

Another mode of visualization actually produced that Figure, however.  The Python script \texttt{c/ch6/plotTS.py} works with \texttt{binary} output from \pTS monitoring, like this:
\begin{cline}
$ ./ode -ts_monitor binary:t.dat -ts_monitor_solution binary:y.dat
$ ./plotTS.py -o figure.png t.dat y.dat
\end{cline}
Running \texttt{plotTS.py} requires copies of, or sym-links to, the Python tools \texttt{PetscBinaryIO.py} and \texttt{petsc\_conf.py}.  (They are in the \texttt{bin/} directory of your \texttt{\$PETSC\_DIR}.)

We are free to adjust the start time, end time, and initial time step at the command line:
\begin{cline}
$ ./ode -ts_init_time 1.0 -ts_final_time 2.0 -ts_dt 0.001 -ts_monitor
\end{cline}
%$
We can also turn off adaptive time-stepping:
\begin{cline}
$ ./ode -ts_adapt_type none -ts_monitor
\end{cline}
%$
(This generates 200 steps using the requested initial step \texttt{dt}$=0.1$.)  On the other hand, we can change the relative tolerance used in adaptive time-stepping from its default value of $10^{-4}$:
\begin{cline}
$ ./ode -ts_monitor -ts_rtol 1.0e-1
$ ./ode -ts_monitor -ts_rtol 1.0e-10
\end{cline}
%$
These runs give 18 and 119 steps, and final numerical errors of $1.2$ and $3.4\times 10^{-3}$, respectively.

We can choose the particular solver type with ease, either the major type with \texttt{-ts\_type} or the Runge-Kutta type with \texttt{-ts\_rk\_type} if the solver is already of type \texttt{rk}:
\begin{cline}
$ ./ode -help |grep ts_type
  -ts_type <rk>: TS method (one of) euler beuler cn pseudo gl ssp theta alpha rk
                 arkimex rosw eimex mimex
$ ./ode -ts_type rk -help |grep ts_rk
  -ts_rk_type <3bs> (choose one of) 5dp 5f 4 3bs 3 2a 1fe
\end{cline}

We encourage the reader to compare \pTS methods on this first ODE example problem; see Exercises \ref{chap:ts}.\ref{exer:ts:odepossible} and \ref{chap:ts}.\ref{exer:ts:odeserial}.  However, we defer comparing methods until after introducing the concepts of stiffness and stability.  Furthermore we compare methods on a much larger ODE system, namely one generated from the time-dependent heat equation, a PDE.

\section{Implicitness, stiffness, and stability}  With the exception of the backward Euler method \eqref{eq:ts:backwardeuler}, all of the methods mentioned above are \emph{explicit}.  They can be written as single formulas which compute $\bY_\ell$ from previous solution values via a fixed number of applications of the right-hand side $\bg$.  For example, the explicit trapezoid rule \eqref{eq:ts:rk2a} can be written
\begin{equation*}
\bY_\ell = \bY_{\ell-1} + \frac{h}{2} \bg(t_{\ell-1},\bY_{\ell-1}) + \frac{h}{2} \bg\big(t_\ell,\bY_{\ell-1} + h\, \bg(t_{\ell-1},\bY_{\ell-1})\big).
\end{equation*}
In terms of their tableau, explicit Runge-Kutta schemes only have non-zero coefficients $a_{ij}$ strictly below the diagonal (Table \ref{tab:ts:exampletableau}).

\emph{Implicit} methods require solving equations.  For instance, the backward Euler method \eqref{eq:ts:backwardeuler} can be written as the (generally) nonlinear system
    $$\bF(\bY_\ell)=0$$
for the $N$ real variables, the components of $\bY_\ell$, with
    $$\bF(\bX) = \bX - \bY_{\ell-1} - h \bg(t_\ell,\bX).$$
We will solve such systems by Newton's method, naturally.  At least one nonlinear solve is required at each time-step.

A family of implicit schemes are the \emph{theta ($\theta$) methods} with parameter $0<\theta\le 1$,
\begin{equation}
\bY_\ell = \bY_{\ell-1} + (1-\theta) h \bg(t_{\ell-1},\bY_{\ell-1}) + \theta h \bg(t_\ell,\bY_{\ell}),  \label{eq:ts:theta}
\end{equation}
which have tableau
\begin{center}
\begin{tabular}{c|cc}
$0$   \\
$1$ & $1-\theta$ & $\theta$ \\ \hline
    & $1-\theta$ & $\theta$
\end{tabular}
\end{center}
The $\theta=0$ case of formula \eqref{eq:ts:theta}, excluded from now on when discussing $\theta$ methods, is the explicit forward Euler method \eqref{eq:ts:forwardeulerupdate}.  The $\theta=1$ case is the backward Euler method \eqref{eq:ts:backwardeuler}.

The Euler methods are first-order but the $\theta=1/2$ case of \eqref{eq:ts:theta}, the \emph{trapezoid} method, or \emph{Crank-Nicolson} to users familiar with PDE methods \citep{MortonMayers2005}, is second-order $O(\Delta t^2)$.  The explicit version of the trapezoid rule \eqref{eq:ts:rk2a} is second-order too.  Why would one go to the effort of solving a system of algebraic equations at each time step?  The answer is that \emph{stability} is obligatory in a numerical scheme while higher-order is merely desirable.  Explicit schemes are stable with sufficiently-small time steps, but taking many small time steps is a computational burden.  Before we can address the trade-off between the computational burden of many time-steps versus that of solving equations, we relate the \emph{stiffness} of certain ODE problems to scheme stability.

As a practical definition,\sidenote{There is no precise mathematical way to decide if a particular ODE system is ``stiff''.  In that sense ``stiff'' is like ``sparse''; one usually appends ``\dots enough so that [something specific happens]'' if one needs to be quantitative.} an ODE system is \emph{stiff} if it has some short time-scales which do not need to be resolved as accurately as certain other time-scales in the problem, but which cause explicit schemes to take short time-steps.  An example is appropriate.

\noindent\hrulefill
\begin{example}  \label{ex:ts:odestiff}  Consider the linear ODE IVP, in $N=3$ dimensions,
\begin{equation}
   \by' = B \by, \qquad \by(0) = \begin{bmatrix} 1 \\ 1 \\ 1 \end{bmatrix} \label{eq:ts:stiffexample}
\end{equation}
where
\begin{equation}
   B = \begin{bmatrix} 0 & 1 & 0 \\
                      -1 & 0 & 0.1 \\
                       0 & 0 & -101 \end{bmatrix}. \label{eq:ts:stiffexamplematrix}
\end{equation}
The exact solution at a particular time, say $t_f=10$ for concreteness, can be computed by exponentiating a matrix,
\begin{equation}
    \by(10) = e^{B t_f} \by(0) = \begin{bmatrix} -1.383623 \\
                                                 -0.295886 \\
                                                  0 \end{bmatrix}.\label{eq:ts:stiffsoln}
\end{equation}
(To compute this conveniently by machine one needs a sufficiently complete set of dense-matrix capabilities.  The above result was calculated with Python, not \PETSc, but it can be done by hand as well; see Exercise \ref{chap:ts}.\ref{exer:ts:stiffexample}.)
\end{example}
\noindent\hrulefill

We have mentioned two $O(\Delta t^2$) methods already, namely the explicit trapezoid rule \eqref{eq:ts:rk2a} and the implicit trapezoid rule which is the $\theta=1/2$ method \eqref{eq:ts:theta}.  Having written a modification of \texttt{ode.c} called \texttt{stiff.c}, as in Exercise \ref{chap:ts}.\ref{exer:ts:stiffexample}, we can compare these methods in the case with no adaptive time-stepping.  Starting with the explicit method, let us take 100 steps from $t_0=0$ to $t_f=10$:
\begin{cline}
$ ./stiff -ts_type rk -ts_rk_type 2a -ts_adapt_type none -ts_dt 0.1
Vec Object: 1 MPI processes
  type: seq
1.65158e+157
-1.66809e+159
1.68494e+162
total steps = 100
\end{cline}
%$
This is nonsense; an explosion.  Using 200 steps is not much better.  However, using 1000 steps gives an answer which is has three-digit accuracy:
\begin{cline}
$ ./stiff -ts_type rk -ts_rk_type 2a -ts_adapt_type none -ts_dt 0.01
Vec Object: 1 MPI processes
  type: seq
-1.38367
-0.295656
1.03141e-301
total steps = 1000
\end{cline}

Compare the implicit method, again without adaptive time-stepping.  For 100 steps we get a reasonable answer and for 100 steps we get about two-digit accuracy:
\begin{cline}
$ ./stiff -ts_type cn -ts_adapt_type none -ts_dt 0.1
Vec Object: 1 MPI processes
  type: seq
-1.38111
-0.307389
3.71505e-18
total steps = 100
$ ./stiff -ts_type cn -ts_adapt_type none -ts_dt 0.05
Vec Object: 1 MPI processes
  type: seq
-1.383
-0.298767
1.6678e-73
total steps = 200
\end{cline}
%$
This comparison of explicit and implicit trapezoid rules, both putatively $O(\Delta t^2)$ methods, suggests something other than local truncation error is important.

FIXME: figure

FIXME: explain goal of \emph{absolute stability}

FIXME: before going on, note small fixed dimension adaptive \RKthreebs does o.k.~but very insensitive to tolerance (second run uses defaults)
\begin{cline}
$ ./stiff -ts_rtol 1.0e-6 -ts_atol 1.0e-6
$ ./stiff -ts_rtol 1.0e-4 -ts_atol 1.0e-4
$ ./stiff -ts_rtol 1.0e-3 -ts_atol 1.0e-3
$ ./stiff -ts_rtol 1.0e-2 -ts_atol 1.0e-2
\end{cline}
%$
gives 430, 411, 405, 403 steps resp.

\section{\pTS support for stiff systems: Jacobians}   FIXME: demonstrate transformation of an implicit time step into form $\bF(\bx)=0$

FIXME \pTS contains \pSNES inside (and thus the rest)

\begin{table}
\small
\begin{tabular}{lllll}
\emph{method}                                & \texttt{-ts\_type} & \emph{adaptive?} & \emph{common options} \\ \hline
forward Euler \eqref{eq:ts:forwardeuler}    & \texttt{euler}  & no & \\
Runge-Kutta  FIXME                          & \texttt{rk}     & yes & \texttt{-ts\_rk\_type} FIXME \\ \hline
backward Euler \eqref{eq:ts:backwardeuler}  & \texttt{beuler} & no & \\
$\theta>0$   FIXME                          & \texttt{theta}  & no & FIXME \\
Crank-Nicolson                              & \texttt{cn}     & no & 
\end{tabular}
\caption{Choices of the best-known ODE integration methods.  Explicit schemes are above the line and implicit below.  Compare Table \ref{tab:ts:tstypes}, which is more complete.} \label{tab:ts:odebasictypes}
\end{table}

FIXME We set a Jacobian call-back to our function \texttt{FormRHSJacobian()} which sets the entries of the \pMat \texttt{J} corresponding to
    $$\frac{\partial \bg}{\partial \by}(t,\by) = \begin{bmatrix}\phantom{\bigg|} \frac{\partial g_i}{\partial y_j}(t,\by) \phantom{\bigg|}\end{bmatrix}.$$
Because an allocated \pMat must be supplied to \texttt{TSSetRHSJacobian()}, the \pTS set-up lines are preceded, in Code \ref{code:ts:ode:main}, by a \texttt{Create/SetUp} sequence for \pMat \texttt{J}.  As usual when setting Jacobians, we assign \texttt{J} to both the Jacobian and preconditioner-material arguments of \texttt{TSSetRHSJacobian()}; compare usage of \texttt{SNESSetJacobian()} and \texttt{DMDASNESSetJacobianLocal()} in Chapters \ref{chap:nl} and \ref{chap:of}.

\cinputpart{odejac.c}{\CODELOC}{FIXME}{I}{//JACOBIAN}{//ENDJACOBIAN}{code:ts:odejac:jacobian}

\cinputpart{odejac.c}{\CODELOC}{FIXME}{II}{//MATJ}{//ENDMATJ}{code:ts:odejac:mat}

FIXME \texttt{odejac.c} is similar to \texttt{ecjac.c}

FIXME build all tools around the ``hard'' time dependent case of stiff + nonlinear, though not DAE

FIXME control \texttt{euler,beuler,cn,theta} by setting steps

FIXME use \texttt{-help} for additional:

\begin{cline}
$ ./ode -ts_monitor -ts_type theta -help |grep ts_theta
  -ts_theta_theta <0.5>: Location of stage (0<Theta<=1)
  -ts_theta_extrapolate: <FALSE> Extrapolate stage solution from previous ...
  -ts_theta_endpoint: <FALSE> Use the endpoint instead of midpoint form of ...
  -ts_theta_adapt: <FALSE> Use time-step adaptivity with the Theta method
\end{cline}
%$

%// ./ode -ts_monitor -ts_type beuler -ode_steps 1000    // finally close to default RK
%// ./ode -log_view |grep Eval   // compare rk, beuler, cn



\section{Time-dependent heat equation}

FIXME solve heat equation by implicit

\cinputpart{heat.c}{\CODELOC}{FIXME}{I}{//HEATCTX}{//ENDHEATCTX}{code:ts:heat:heatctx}

\cinputpart{heat.c}{\CODELOC}{FIXME}{II}{//RHSFUNCTION}{//ENDRHSFUNCTION}{code:ts:heat:rhsfunction}

\cinputpart{heat.c}{\CODELOC}{FIXME}{III}{//RHSJACOBIAN}{//ENDRHSJACOBIAN}{code:ts:heat:rhsjacobian}

\cinputpart{heat.c}{\CODELOC}{FIXME}{IV}{//TSSETUP}{//ENDTSSETUP}{code:ts:heat:tssetup}

\cinputpart{heat.c}{\CODELOC}{FIXME}{V}{//MONITOR}{//ENDMONITOR}{code:ts:heat:monitor}

\begin{figure}
% usage: \standardTSstack{scale}{Jacobian}{DMDA}{DMPlex}
\standardTSstack{0.775}{}{}{dashed}
\caption{The \PETSc stack used for the time-dependent PDE problems in Chapter \ref{chap:ts} (i.e.~\texttt{heat.c,pattern.c}).  Compare Figure \ref{fig:of:standardstack}.}
\label{fig:of:tsstack}
\end{figure}

FIXME show \texttt{-ts\_type theta,beuler,cn}


\section{Coupled reaction-diffusion equations}

FIXME see pages 21--22 of \citep{HundsdorferVerwer2003} and see \citep{Pearson1993}

\cinputpart{pattern.c}{\CODELOC}{FIXME}{I}{//FIELDCTX}{//ENDFIELDCTX}{code:ts:pattern:fieldctx}

\cinputpart{pattern.c}{\CODELOC}{FIXME}{II}{//RHSFUNCTION}{//ENDRHSFUNCTION}{code:ts:pattern:rhsfunction}

\cinputpart{pattern.c}{\CODELOC}{FIXME}{III}{//IFUNCTION}{//ENDIFUNCTION}{code:ts:pattern:ifunction}

\cinputpart{pattern.c}{\CODELOC}{FIXME}{IV}{//IJACOBIAN}{//ENDIJACOBIAN}{code:ts:pattern:ijacobian}

\cinputpart{pattern.c}{\CODELOC}{FIXME}{V}{//TSSETUP}{//ENDTSSETUP}{code:ts:pattern:tssetup}

FIXME show use of \texttt{-ts\_type arkimex,theta}

FIXME final-time greyscale plots using \texttt{PetscBinaryIO.py}


\section{Exercises}

\renewcommand{\labelenumi}{\arabic{chapter}.\arabic{enumi}\quad}
\renewcommand{\labelenumii}{(\alph{enumii})}
\begin{enumerate}
\item \label{exer:ts:tan}  Consider the scalar ODE initial value problem $y'=1+y^2$, $y(0)=0$.  Show by-hand that $y(t)=\tan t$ is the (unique) solution.  Modify \texttt{ode.c} to solve this problem.  Run the code from $t_0=0$ to $t_f=2$.  What run-time evidence shows that your estimate of ``$y(2)$'' is totally meaningless?

\item \label{exer:ts:odepossible} Which \pTS types work with \texttt{ode.c}, that is, do not give run-time errors?  Do \texttt{-help |grep ts\_type} to find possibilities, and try them.  Which ones work with the additional option \texttt{-snes\_fd}?  Which ones work with \texttt{odejac.c}?  Explain as much as you can.

\item \label{exer:ts:odeserial} Code \texttt{ode.c} claims it is ``serial only.''  What happens with \texttt{mpiexec -n 2 ./ode}, and why?

\item \label{exer:ts:stiffexample}
  \begin{enumerate}
  \item Modify \texttt{odejac.c} to a similar code \texttt{stiff.c} which solves ODE IVP \eqref{eq:ts:stiffexample}.  It can use \texttt{TSGetTotalSteps()} to print the number of steps and \texttt{VecView()} to print the computed solution.  Confirm the results shown in the text for RK methods \texttt{-ts\_rk\_type 2a,3bs} and for \texttt{-ts\_type cn}.
  \item Show by hand that the eigenvalues of matrix $B$ in \eqref{eq:ts:stiffexamplematrix} are $\lambda=i,-i,-101$.  Use a full eigen-decomposition $B = X \Lambda X^{-1}$, which can be computed by hand, to confirm \eqref{eq:ts:stiffsoln}, noting that
    $$e^{Bt} = X e^{\Lambda t} X^{-1} = X \begin{bmatrix} \lambda_0 &  &  \\
                       & \ddots &  \\
                       & & \lambda_N \end{bmatrix} X^{-1}.$$
  \end{enumerate}

\item FIXME version of \texttt{ode.c} which solves DAE system
    $$\bbf(t,\by,\by') = \bg(t,\by)$$
where $\partial \bg/\partial \by'$ may be singular

\item FIXME for run like
\begin{cline}
./pattern -da_refine 6 -ptn_tf 500 -ptn_steps 100 -ts_monitor -snes_converged_reason
\end{cline}
which is fastest among these nine ARKIMEX and three $\theta$ methods?:
\begin{code}
-ts_type arkimex -ts_arkimex_type [a2|l2|ars122|2c|2d|2e|3|4|5]
-ts_type theta -ts_theta_endpoint -ts_theta_theta [0.5|0.75|1]
\end{code}
Are any other adaptive \pTS types faster?  Also compare final frames from \texttt{-ts\_monitor\_solution draw}; how worried should we be about the effect of time-stepping on the solution of this problem?
\end{enumerate}

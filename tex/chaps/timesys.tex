
To the reader who has become lost in the maze of issues which have built up around the examples so far---Krylov solvers, preconditioner choices, line searches, finite element details, or whatever causes confusion---some good news.  In \emph{this} Chapter we restart with an easy question and a small example.

After all, solving \emph{ordinary} differential equations (ODEs) with \PETSc ought to be easy.  Given that we have practice with \PETSc objects, it is.  So is making the transition to numerical methods for time-dependent, parabolic PDEs like the heat equation.  In fact, in this Chapter we look at three examples:
\begin{itemize}
\item a linear system of two ODEs,
\item an arbitrarily-large system of ODEs arising from spatial finite-difference approximation of the time-dependent heat PDE,
\item and another large ODE system arising from two coupled nonlinear PDEs which form a two-species reaction/diffusion model.
\end{itemize}

These time-dependent problems are solved using a new \PETSc object, a \pTS time-stepping solver, but the particular integration method can be chosen at run-time.  We will recall the basics of both \emph{explicit} and \emph{implicit} time-stepping techniques.  Because an implicit choice implies the full stack of \pSNES/\pKSP/\pPC solvers for the (generally) non-linear equations at each time step, our experience with examples from Chapters \ref{chap:st}--\ref{chap:of} will all be useful in the implicit case.

Our \emph{method of lines} approach for the two PDE examples above avoids any need to write code specific to the temporal discretization of those PDEs.  The third example above is, however, the first \emph{system} of PDEs addressed so far.  Describing such a system using \PETSc calls requires new choices in the code.


\section{ODEs and \PETSc \pTS objects}

Consider an ODE system in form
\begin{equation}
\by' = \bg(t,\by)  \label{eq:ts:ode}
\end{equation}
where $\by(t) \in \RR^N$ and $\by'=\by'(t)$ denotes $d\by/dt$, and suppose
\begin{equation}
\by(t_0) = \by_0  \label{eq:ts:ode:iv}
\end{equation}
gives an \emph{initial value}.

It is well-known that under reasonable assumptions on the behavior of $g$, the continuum problem \eqref{eq:ts:ode}, \eqref{eq:ts:ode:iv} is a well-posed problem for at least a short time away from $t_0$.  That is, one can make precise predictions forward (or backward) from the initial time, supposing that the integration to solve the ODE is performed exactly.

Let us be specific about sufficient regularity of $\bg$ to imply well-posedness.  We first assume for simplicity that $\bg(t,\by)$ is continuous a product domain (cylinder) around the initial point $(t_0,\by_0)$,
   $$\mathcal{D} = \{(t,\by) \,:\, |t-t_0| \le \delta, \|\by - \by_0\| \le \omega\}$$
where $\delta > 0$ and $\omega > 0$.  Then we further assume that $\bg$ is \emph{Lipshitz} in its second argument, so that output differences are bounded by a multiple of input differences: there is $L\ge 0$ so that
   $$\|\bg(t,\by_1) - \bg(t,\by_0)\| \le L \|\by_1-\by_0\|$$
for $(t,\by_i) \in \mathcal{D}$.  Then the problem \eqref{eq:ts:ode}, \eqref{eq:ts:ode:iv} has a unique continuous, and continuously-differentiable, solution $\by(t)$ on a possibly-shorter interval ($|t-t_0|<\eps$ for some $0 < \eps \le \delta$) \citep[section 17.5]{HirschSmaleDevaney2004}.  Furthermore the solution depends continuously on both the initial value $\by_0$ and the right-hand-side $\bg$.

Our attention to well-posedness is motivated by the following practical concern:  When we run a differential-equation-solving code it produces numbers.\sidenote{It \emph{always} produces numbers!}  These numbers are (essentially) never the exact solution of the differential equation, but they are ``correct'' in a numerical-analysis sense if we can demonstrate convergence to the solution of a well-posed continuum problem.  \emph{Much} more egregious, however, is to produce numbers which represent no continuum solution at all.  Benign-looking scalar nonlinear ODEs can put us in such peril.  In fact, Exercise \ref{chap:ts}.\ref{exer:ts:tan} gives a well-known example where the solution ceases to exist after a certain finite interval of time, though the function $g(t,y)$ is merely a quadratic function of $y$.  Because the scope of the current book includes many nonlinear PDEs, where this concern is yet greater, observing the dangers lurking in ODE problems is appropriate.

The following example system of two ODEs generates no such concern with well-posedness.  In fact, for linear systems, including those with continuous coefficients and non-homogeneities, solutions exist for all time \citep[section 17.5]{HirschSmaleDevaney2004}.

\noindent\hrulefill
\begin{example}  Consider the initial value problem in two dimensions
\begin{equation}
   \by' = \begin{bmatrix} y_1 \\ - y_0 + t\end{bmatrix}, \qquad \by(0) = \begin{bmatrix} 0 \\ 0 \end{bmatrix}, \label{eq:ts:example}
\end{equation}
that is, the linear system with
    $$\bg(t,\by) = A \by + \bbf \,\text{ where } A = \begin{bmatrix} 0 & 1 \\ -1 & 0 \end{bmatrix} \text{ and } \bbf(t) = \begin{bmatrix} 0 \\ t\end{bmatrix}.$$
FIXME solution known
\end{example}

FIXME introduce \pTS, which contains \pSNES inside (and thus the rest)

FIXME function $\bg$ is \texttt{TSSetRHSFunction()}, Jacobian $\partial \bg/\partial \by$ is \texttt{TSSetRHSJacobian()}

FIXME show \texttt{ode.c}, which is most similar to \texttt{ecjacobian.c}

\vfill
\clearpage

\cinputpart{ode.c}{\CODELOC}{FIXME}{I}{//CALLBACKS}{//ENDCALLBACKS}{code:ts:ode:callbacks}

\cinputpart{ode.c}{\CODELOC}{FIXME}{II}{//MAIN}{//ENDMAIN}{code:ts:ode:main}

FIXME intro numerical esp RK and theta \citep{AscherPetzold1998}

FIXME build all tools around the ``hard'' time dependent case of stiff + nonlinear, though not DAE

FIXME control \texttt{euler,beuler,cn,theta} by setting steps

FIXME control \texttt{rk} by \texttt{-ts\_type rk -ts\_atol Z -ts\_rk\_type XX}

FIXME generate figures by, and explain usage, \texttt{PetscBinaryIO.py}

\section{Time-dependent heat equation}

FIXME solve heat equation by implicit

\cinputpart{heat.c}{\CODELOC}{FIXME}{I}{//HEATCTX}{//ENDHEATCTX}{code:ts:heat:heatctx}

\cinputpart{heat.c}{\CODELOC}{FIXME}{II}{//RHSFUNCTION}{//ENDRHSFUNCTION}{code:ts:heat:rhsfunction}

\cinputpart{heat.c}{\CODELOC}{FIXME}{III}{//RHSJACOBIAN}{//ENDRHSJACOBIAN}{code:ts:heat:rhsjacobian}

\cinputpart{heat.c}{\CODELOC}{FIXME}{IV}{//TSSETUP}{//ENDTSSETUP}{code:ts:heat:tssetup}

\cinputpart{heat.c}{\CODELOC}{FIXME}{V}{//MONITOR}{//ENDMONITOR}{code:ts:heat:monitor}

\begin{figure}
% usage: \standardstack{scale}{objective}{Jacobian}{TS}{DMDA}{DMPlex}
\standardstack{0.775}{dashed}{}{}{}{dashed}
\caption{The \PETSc stack used for the time-dependent heat and reaction/diffusion problems (\texttt{heat.c,pattern.c}).  Compare Figure \ref{fig:of:standardstack}.  FIXME: \pSNES not directly seen by user code}
\label{fig:of:tsstack}
\end{figure}

FIXME show \texttt{-ts\_type theta,beuler,cn}


\section{Coupled reaction-diffusion equations}

FIXME see pages 21--22 of \citep{HundsdorferVerwer2003} and see \citep{Pearson1993}

\cinputpart{pattern.c}{\CODELOC}{FIXME}{I}{//FIELDCTX}{//ENDFIELDCTX}{code:ts:pattern:fieldctx}

\cinputpart{pattern.c}{\CODELOC}{FIXME}{II}{//RHSFUNCTION}{//ENDRHSFUNCTION}{code:ts:pattern:rhsfunction}

\cinputpart{pattern.c}{\CODELOC}{FIXME}{III}{//IFUNCTION}{//ENDIFUNCTION}{code:ts:pattern:ifunction}

\cinputpart{pattern.c}{\CODELOC}{FIXME}{IV}{//IJACOBIAN}{//ENDIJACOBIAN}{code:ts:pattern:ijacobian}

\cinputpart{pattern.c}{\CODELOC}{FIXME}{V}{//TSSETUP}{//ENDTSSETUP}{code:ts:pattern:tssetup}

FIXME show use of \texttt{-ts\_type arkimex,theta}

FIXME final-time greyscale plots using \texttt{PetscBinaryIO.py}


\section{Exercises}

\renewcommand{\labelenumi}{\arabic{chapter}.\arabic{enumi}\quad}
\renewcommand{\labelenumii}{(\alph{enumii})}
\begin{enumerate}
\item \label{exer:ts:tan}  Consider the scalar ODE initial value problem $y'=1+y^2$, $y(0)=0$.  Show by-hand that $y(t)=\tan t$ is the unique solution to this problem.  Modify \texttt{ode.c} to solve this problem.  Run the code from $t=0$ to $t=t_f=2$.  What run-time observable, actual evidence shows that your estimate of ``$y(2)$'' is meaningless?
\item FIXME which \texttt{-ts\_type} work with \texttt{ode.c}?
% euler, beuler, rk, theta, cn
\item FIXME \texttt{ode.c} says it is ``serial only''  what happens with \texttt{mpiexec -n 2 ./ode}, and why?
\item FIXME version of \texttt{ode.c} which solves DAE system
    $$\bbf(t,\by,\by') = \bg(t,\by)$$
where $\partial \bg/\partial \by'$ may be singular
\item FIXME for run like
\begin{cline}
./pattern -da_refine 6 -ptn_tf 500 -ptn_steps 100 -ts_monitor -snes_converged_reason
\end{cline}
which is fastest among these nine ARKIMEX and three $\theta$ methods?:
\begin{code}
-ts_type arkimex -ts_arkimex_type [a2|l2|ars122|2c|2d|2e|3|4|5]
-ts_type theta -ts_theta_endpoint -ts_theta_theta [0.5|0.75|1]
\end{code}
Are any other adaptive \pTS types faster?  Also compare final frames from \texttt{-ts\_monitor\_solution draw}; how worried should we be about the effect of time-stepping on the solution of this problem?
\end{enumerate}

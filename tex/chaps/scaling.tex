
\section{Weak and strong scaling}

FIXME: define weak and strong scaling

FIXME: define arithmetic intensity

\section{Assessing our performance with event-logging and \texttt{-log\_summary}}

\vspace{4cm}

FIXME: also we can put a structured grid in the unstructured code

\begin{marginfigure}
\input{tmp/mesh.1.tikz}
\caption{A structured triangulation of the unit square with $K=32$ triangles and $N=25$ nodes.  The entire boundary is Dirichlet in the problem we consider.}
\label{fig:structuredfem}
\end{marginfigure}

FIXME: use CG and MG, and show cost of assembly

\section{Efficiency over mere scaling}

FIXME: how to show?

\section{Ideas}

FIXME: an idea that is most relevant to nonlinear problems: experimentation with linear solvers (i.e.~inside Newton) is obligatory because examples of linear systems can be found so that any solver comes out faster than any other \citep{Nachtigaletal1992} and examples of linear systems can be found so that well-known Krylov solvers like GMRES can be made to converge at any rate \citep{Greenbaumetal1996}

FIXME: perhaps discuss 64 bit PetscInt for large problems

FIXME: perhaps discuss PetscReal of \verb|__float128| for problems with large condition number where 64 bit \texttt{double} is not enough

\section{Weak and strong scaling}

FIXME: define weak and strong scaling

\section{Assessing our performance with event-logging and \texttt{-log\_summary}}

\vspace{4cm}

FIXME: also we can put a structured grid in the unstructured code

\begin{marginfigure}
\input{mesh.1.tikz}
\caption{A structured triangulation of the unit square with $K=32$ triangles and $N=25$ nodes.  The entire boundary is Dirichlet in the problem we consider.}
\label{fig:structuredfem}
\end{marginfigure}

FIXME: use CG

\section{Classical preconditioners}

FIXME:  ILU?  block Jacobi and GS?

FIXME: somewhere describe Nachtigal et al 1992 result that KSP are strictly inequivalent

\begin{comment}
> Hi all,
>
> Is weighted Jacobi available as a preconditioner ? I can't find it in the
> list of preconditioners. If not, what is the rationale between this choice
> ? It is pretty straightforward to code, so if it is not available I can do
> it without problem I guess, but I am just wondering. In the matrix-free
> case where SOR is not available by default, it may be better than pure
> Jacobi, and much easier to parallelize than SOR.
>
>   Timothee Nicolas

I believe what you are looking for is defined by the following options
  -ksp_type richardson
  -ksp_richardson_scale <value>
  -pc_type jacobi

Thanks,
  Dave May
\end{comment}

\section{Domain-decomposition preconditioners}

FIXME: ASM?

\section{Multigrid preconditioners}

FIXME: just show what we have already

FIXME: perhaps add 3D poisson

\section{Ideas}

FIXME: an idea that is most relevant to nonlinear problems: experimentation with linear solvers (i.e.~inside Newton) is obligatory because examples of linear systems can be found so that any solver comes out faster than any other \citep{Nachtigaletal1992} and examples of linear systems can be found so that well-known Krylov solvers like GMRES can be made to converge at any rate \citep{Greenbaumetal1996}

%\caveat{But real-world PDEs are nonlinear.}
